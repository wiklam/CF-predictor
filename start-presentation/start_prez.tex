%TODO
%Wystąpienie w piątek

\documentclass{beamer}
\usetheme{Warsaw}
\setbeamertemplate{navigation symbols}{}

\usepackage{amsthm}
\usepackage{amssymb}
\usepackage{amsmath}
\usepackage{mathrsfs}
\usepackage{mathtools}

\usepackage{algorithm}
\usepackage{algorithmicx}
\usepackage[noend]{algpseudocode}
\usepackage{algpseudocode}

\usepackage{tikz}
\usepackage{tkz-graph}
\usetikzlibrary{matrix}
\usetikzlibrary{arrows.meta}

\usepackage{array}
\usepackage{tabularx}
\usepackage[utf8]{inputenc}

\algnewcommand{\algorithmicand}{\textbf{ and }}
\algnewcommand{\algorithmicor}{\textbf{ or }}
\algnewcommand{\OR}{\algorithmicor}
\algnewcommand{\AND}{\algorithmicand}
\newcommand{\Oh}{\mathcal{O}}
\newcommand{\Ohtilda}{\tilde{\Oh}}
\newcommand{\eps}{\varepsilon}
\algnewcommand{\LineComment}[1]{\State \(\triangleright\) #1}

\newcommand{\countelem}{\textsf{cnt}}

\title[Machine Learning]{CF Predictor}
\subtitle{Rating Predictor for Codeforces}
\author[CF Predictor]{Anadi Agrawal, Hubert Obrzut, Wiktor Pilarczyk i Michał Syposz}
\institute{Uniwersytet Wrocławski}
\date{\today}

\begin{document}

\begin{frame}
\titlepage
\end{frame}



\begin{frame}
\frametitle{Why codeforces?}

\begin{itemize}
	\item<1-> The world's largest platform for programming;
	\item<2-> A lot of data to process;
	\item<3-> Great access to data via Codeforces API;
	\item<4-> There is a need for such a tool for this website.
\end{itemize}

\end{frame}





\begin{frame}
\frametitle{Rating predictor}

\begin{exampleblock}{Codeforces's rating}
Codeforces uses rating's system is somehow similar to the ELO used in chess. 
Once for a while, Codeforces hosts the round after which the rating is recalculated based on results achieved by each participant.
\end{exampleblock}

\onslide<2->{
\begin{block}{What do we want to do?}
	We want to be able to predict the outcome of the round, i.e. before the round, for a specific user, to predict his ratings' change based on specific factors:
	\begin{itemize}
		\item<3-> Current form of the participant;
		\item<4-> The authors of this round;
		\item<5-> The time when the round is hosted;
		\item<6-> The length of the round.
	\end{itemize}
\end{block}}

\end{frame}



\end{document}